%%% CHAPTER Introduction ------------------------------begin----
\chapter{Introduction}
\section{Motivation}
Developing and distributing effective software is one of the most important concerns of today's software-driven fields. Effective software is surely needed in almost every part of embedded systems, especially in the fields of automotive, robotics, defense, transportation, electrical instruments, autonomous and cyber-physical systems. Quality software involvement in fields such as the ones that are mentioned created a great demand for parallel software development in the last ten years. This great demand caused software engineers in especially IT and embedded system sector to study parallel computing along with multi- and many- core systems.

The digitalization of almost every aspect of our lives as we know it requires systems to be more and more complex each passing day. While some years ago the computers had single-core processors, today almost every single computer has at least a couple of cores in their processors. The advancements in processors allowed development of more advanced systems with efficient software. The super computers currently NASA (The National Aeronautics and Space Administration) uses for collecting information are said to record as much data as it has been collected in the entire the world history in just four years. This example should show that how complex applications can be in the century we are living in. Furthermore, one of the most trending topics Cloud Computing, which is being studied to make use of complex computing power of super computers' remotely to public users, is being researched and it will benefit greatly from the advancements in the field of parallel computing.

While parallel computing is used for achieving more complex software, it is also widely used in more basic and cheap processors in order to achieve more tasks with less resource consumption and cost. This is achieved by proper scheduling techniques. Furthermore, with an efficient software distributed efficiently to a processor's cores, one could also make use of less energy consumption features by applying techniques such as under-clocking a processor. To summarize, developing efficient parallel software is not only useful in achieving advanced computing capability but also can help to achieve less energy and resource consumption, thus decreasing the cost of systems and making them more environment-friendly.

\section{Objective}
Even though achieving concurrency using parallel computing is crucial, it might lead to error-prone systems if software is not planned and executed properly. Developers have to consider using the right software and also have to determine and plan not only the hardware constraints but also the software constraints in order to create an efficient and reliable software.

Before its execution, parallel software have to be delicately planned. The first stage of the parallel software development, planning stage, involves several activities such as Modeling, Partitioning, Task generation and Mapping. In the modeling stage, hardware and software model needs to be created. While software model is described by defining runnables, labels, label accesses, runnable activations and software constraints; the hardware model is described by defining processor details, hardware system clock and core information. After the modeling activity, partitioning is done that determines which group of runnables belong together. Partitioning results are combined with system constraints in order to generate tasks. Final activity, Mapping, involves laying out the details about pinning generated tasks to available hardware units and their cores.

While there are some commercial tools that provide easement in the parallel software development, recent study done in Germany, namely AMALTHEA4public \cite{ICPDSSE} \cite{amalthea4publicweb}, aims to provide planning and tracing tools especially for multi-core developments in automotive domain with several open source development tools. The branch of AMALTHEA4public, APP4MC project \cite{app4mcproposaleclipse} provides an Eclipse-based tool chain environment and de-facto standard to integrate tools for all major design steps in the multi- and many-core development phase. A basic set of tools are available to demonstrate all the steps needed in the development process. The APP4MC project aims at providing \cite{app4mcproposaleclipse}:

\begin{itemize}
	\item A basis for the integration of various tools into a consistent and comprehensive tool chain.
	\item Extensive models for timing behaviour, software, hardware, and constraints descriptions (used for simulation / analysis and for exchange).
	\item Editors and domain specific languages for the models.
	\item Tools for scheduling, partitioning, and optimizing of multi- and many-core architectures \cite{app4mcproposaleclipse}.
\end{itemize}

The author's aim with this project is to investigate and evaluate APP4MC's performance with real-world distributed system in several aspects such as core utilization, energy consumption and resource usage while studying efficient parallel computing activities at his time with Project AMALTHEA4public.

\section{Methodology}
in this paper..

all functionalities and autonomy, cool features..
to simulate an automotive domain application

other chapters explain briefly.



%%% CHAPTER Introduction ------------------------------end----
%%% CHAPTER Multi-core Programming ------------------begin----
\chapter{Multi-core Programming}

%%% CHAPTER Multi-core Programming ------------------end----
%%% CHAPTER APP4MC Development Environment ----------begin----
\chapter{APP4MC Development Environment}

%%% CHAPTER APP4MC Development Environment ------------end----
%%% CHAPTER A4MCAR ----------------------------------begin----
\chapter{Distributed Heterogeneous Demonstrator System (A4MCAR) Design}

%%% CHAPTER A4MCAR ------------------------------------end----
%%% CHAPTER Effective Parallelism Evaluation --------begin----
\chapter{Effective Parallelism Evaluation}

%%% CHAPTER Effective Parallelism Evaluation --------end----
%%% CHAPTER Conclusion ----------------------------begin----
\chapter{Conclusion}

%%% CHAPTER Conclusion ------------------------------end----