%\textbf{Wichtig:} Das hier vorgestellte Layout ist nicht verpflichtend. Es spiegelt das pers�nliche Empfinden des Autors wieder und kann den eigenen Bed�rfnissen entsprechen angepasst und erweitert werden. Das Dokument soll lediglich einen einfachen Einstieg in \LaTeX  zum Erstellen von Abschlussarbeiten erm�glichen und zus�tzlich Hilfestellung zum wissenschaftlichen Arbeiten geben.\\

\section*{Zusammenfassung}
\label{sec:Zusammenfassung}
Verteilung von Software effektiv an Multi wurde Core, viele Kern und verteilten Systemen untersucht seit Jahrzehnten aber noch Fortschritte nacheinander angetrieben von Domain-spezifische Einschr�nkungen. 
Programmierung Fahrzeug ECU ist eines der am meisten eingeschr�nkten Dom�nen, die k�rzlich die Notwendigkeit einer Parallelit�t durch fortschrittliche Fahrerassistenzsysteme oder autonome treibende Ans�tze n�herte. 

In diesem Papier Software-Verteilung-Herausforderungen f�r solche Systeme werden diskutiert und L�sungen werden auf Anweisung pr�zise Modellierung, Affinit�t eingeschr�nkt Verteilung und Verringerung der Aufgabe Reaktionszeiten erreicht durch fortschrittliche Software Parallelisierung vorgestellt. Daher sind APP4MCs Partitionierung und mapping-Algorithmen avancierte zum Affinit�t Zw�nge, Software-Betriebsmittel-Kennzeichnungen und Kommunikationskosten ber�cksichtigen.

Eine Demonstrator-System namens A4MCAR entwickelt wurde, die nicht nur niedrige Ebene Funktionalit�ten wie Sensor und motor fahren aber auch hohe Level-Features wie Bildverarbeitung, verf�gt �ber Kamera streaming, Server-basierte drahtlose fahren �ber Internet, Bluetooth-Anbindung via Android-Anwendung, System monitoring und Analyse Kernfunktionen und Touchscreen Benutzeroberfl�che. Unsere Experimente entlang der heterogenen Multi-Task-Demonstrator A4MCAR zeigen, dass mit APP4MC Ergebnisse anstelle von OS-basierten oder sequentielle Implementierungen auf einem verteilten heterogenen System deutlich verbessert ihre Reaktionsf�higkeit um potenziell Energieverbrauch zu verringern und Fehler anf�llig manuelle Einschr�nkung �berlegungen f�r gemischt-kritische Anwendungen ersetzt.


\section*{Abstract}
\label{sec:Abstract}
Distributing software effectively to multi core, many core, and distributed systems has been studied for decades but still advances successively driven by domain specific constraints. 
Programming vehicle ECUs is one of the most constrained domains that 
recently approached the need for concurrency due to advanced driver assistant systems or autonomous driving approaches. 

In this paper, software distribution challenges for such systems are discussed and solutions are presented upon instruction precise modeling, affinity constrained distribution, and reducing task response times achieved by advanced software parallelization.  Therefore, APP4MC's partitioning and mapping algorithms are advanced to consider affinity constraints, software component tags and communication costs.

A demonstrator system called A4MCAR has been developed which features not only low level functionalities such as sensor and motor driving but also high level features such as image processing, camera streaming, server-based wireless driving via Web, bluetooth connectivity via Android application, system core monitoring and analysis features and touchscreen UI. Our experiments along the multi-task heterogeneous demonstrator A4MCAR show that using APP4MC results instead of OS-based or sequential implementations on a distributed heterogeneous system significantly improves its responsiveness in order to potentially reduce energy consumption and replaces error prone manual constraint considerations for mixed-critical applications.  